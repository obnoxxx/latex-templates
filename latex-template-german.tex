%%%================%%%
%%%  LaTeX-Ger�st  %%%
%%%================%%%

%%
%%  Nach Bedarf auskommentieren oder Kommentare entfernen
%%



%%
%% Genau eine Dokumentenklasse w�hlen:
%%
\documentclass[12pt,a4paper]{article}
%\documentclass[11pt,a4paper]{article}
%\documentclass[10pt,a4paper]{article}
%
%\documentclass[12pt,a4paper]{book}
% 
%\documentclass{letter}
%...


\usepackage{german}      % Deutsche TeX-Eigenheiten
\usepackage{isolatin1}   % Eingabekodierung mit Umlauten...


%%
%% Index-Erstellung
%%
\usepackage{makeidx}
\makeindex            % damit eine Indexdatei angelegt wird


%%
%% AMS-LaTeX Pakete:
%%
\usepackage{amsmath}  % allgemeine Mathe-Erweiterungen
\usepackage{amssymb}  % Symbole und Schriftarten
\usepackage{amsthm}   % erweiterte Theorem-Umgebungen


%%
%% XY-Pic f�r Diagramme etc
%%
%\usepackage[all]{xy}      % Das Paket mit allem, was man so braucht
%\UseComputerModernTips    % Pfeilspitzen wie im normalen Mathe-Modus
%\CompileMatrices          % Damit geht es etwas schneller.


\usepackage{mathrsfs}  % gibt den Befehl "\mathscr{}" f�r sch�ne
                       % Mathe-Skript-Buchstaben




%%
%% F�r den Titel:
%%
%\author{G. Harder}
%\title{}
%\date{}   % Standard: Datum der Kompilierung ("\today").


%%%--------------------------%%%
%%%  Gleich geht es los...   %%%
%%%--------------------------%%%
\begin{document}
%%
%% Titel:
%%
%% * Entweder Standardtitel mit obigen "\title" und "\author"
%\maketitle
%% * oder alternativ eine frei formatierte Titelseite:
%\begin{titlepage}
%\end{titlepage}

%%
%% Zusammenfassung:
%%
%\begin{abstract}
%\end{abstract}

%%
%% Inhaltsverzeichnis:
%%
%\tableofcontents

%%%-------------------------------%%%
%%%  Jetzt geht es richtig los:   %%%
%%%-------------------------------%%%










%%%----------------------------%%%
%%%  Jetzt geht es zu Ende...  %%%
%%%----------------------------%%%

%%
%%  Bibliographie
%%




%%
%% Stichwortverzeichnis:
%%
%\renewcommand{\indexname}{Stichworte}  % Soll der "Index" anders hei�en?
%\printindex                            % Stichwortverzeichnis ausgeben.

\end{document}
%%%%%%%%%%%%%%
%%%  Ende  %%%
%%%%%%%%%%%%%%


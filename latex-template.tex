%%%================%%%
%%% LaTeX templat %%%
%%%================%%%

%%
%% ddjust and remove cpomments as needed.
%%



%%
%% Genau eine Dokumentenklasse w�hlen: 
%% first, choose exactly one document class.
%% Here are several examples:
%%
\documentclass[12pt,a4paper]{article}
%\documentclass[11pt,a4paper]{article}
%\documentclass[10pt,a4paper]{article}
%
%\documentclass[12pt,a4paper]{book}
% 
%\documentclass{letter}
%...

%% language- and encoding-related packages:
%%
%\usepackage{german}
%\usepackage[utf8]{inputenc}


%%
%% index generation:

\usepackage{imakeidx}
\makeindex


%% for typesetting math:
%% AMS-LaTeX packages:
%%
\usepackage{amsmath}%general math extensions
\usepackage{amssymb}% symbols and fonts
\usepackage{amsthm}% extended theorem environments


%%
%% XY-Pic  for diagrams etc.
%%
%\usepackage[all]{xy}      % Dall of XY-Pic that us usually needed
%\UseComputerModernTips    %arrow tips like in normal math mode
%\CompileMatrices          %  speed up rendering.

\usepackage{mathrsfs}  % provides'\mathscr{}' for nice
                       % math script letters.




%%
%%  set up the title:
%%
%\author{J. Doe}
%\title{}
%\date{}   % Standard: Datum der Kompilierung ("\today").


%%%--------------------------%%%
%%%  Gleich geht es los...   %%%
%%%--------------------------%%%
\begin{document}
%%
%% Titel:
%%
%% * Entweder Standardtitel mit obigen "\title" und "\author"
%\maketitle
%% * oder alternativ eine frei formatierte Titelseite:
%\begin{titlepage}
%\end{titlepage}

%%
%% Zusammenfassung:
%%
%\begin{abstract}
%\end{abstract}

%%
%% Inhaltsverzeichnis:
%%
%\tableofcontents

%%%-------------------------------%%%
%%%  Jetzt geht es richtig los:   %%%
%%%-------------------------------%%%










%%%----------------------------%%%
%%%  Jetzt geht es zu Ende...  %%%
%%%----------------------------%%%

%%
%%  Bibliographie
%%




%%
%% Stichwortverzeichnis:
%%
%\renewcommand{\indexname}{Stichworte}  % Soll der "Index" anders hei�en?
%\printindex                            % Stichwortverzeichnis ausgeben.

\end{document}
%%%%%%%%%%%%%%
%%%  Ende  %%%
%%%%%%%%%%%%%%

